\documentclass[a4paper, 12pt]{article}
\usepackage[T2A]{fontenc}  
\usepackage[utf8]{inputenc} 
\usepackage[english,russian]{babel} 
\usepackage{color}    
\usepackage{float}

\usepackage[top=20mm, bottom=20mm, left=30mm, right=15mm]{geometry}

\usepackage{amsmath, amsfonts, amssymb, mathtools} 

\usepackage{amsthm}

\theoremstyle{definition}
\newtheorem{defn}{Определение}[section]
\newtheorem*{rem}{Замечание}
\usepackage{graphicx}

\graphicspath{{img/}{../../graphics/}}

% Математическое ожидание\
\newcommand{\Expect}{%
	\mathsf{M}}

\renewcommand{\Variance}{%
	\mathsf{D}}

% Математическое ожидание
\newcommand{\m}[1]{%
	\ensuremath{M\!#1}}

% Математическое ожидание X
\newcommand{\mx}{%
	\ensuremath{M\!X}}

\newcommand{\mxx}{%
	\ensuremath{M\!X^2}}

% Математическое ожидание Y
\newcommand{\my}{%
	\ensuremath{M\!Y}}

\newcommand{\myy}{%
	\ensuremath{M\!Y^2}}

% Дисперсия
\newcommand{\disp}[1]{%
	\ensuremath{D\!#1}}

% Дисперсия X
\newcommand{\dx}{%
	\ensuremath{D\!X}}

% Дисперсия Y
\newcommand{\dy}{%
	\ensuremath{D\!Y}}

% Следовательно
\newcommand{\Rarrow}{%
	\ensuremath{\;\Rightarrow\;}}

% Бесконечная последовательность 
\newcommand{\infseq}[3]{%
	\ensuremath{#1_#2, \dots, #1_#3, \dots}\ }

% Бесконечная последовательность X_1, ... X_n, ...
\newcommand{\infseqX}{%
	\infseq{X}{1}{n}}

\usepackage{listings, listingsutf8}

\lstset{
	language = Matlab,
	numbers=left,   
	frame=single,    
	basicstyle=\small,
    escapebegin=\begin{russian}\commentfont,
    escapeend=\end{russian},
    breaklines=true,   
    breakatwhitespace=true,
	literate={а}{{\selectfont\char224}}1
	{б}{{\selectfont\char225}}1
	{в}{{\selectfont\char226}}1
	{г}{{\selectfont\char227}}1
	{д}{{\selectfont\char228}}1
	{е}{{\selectfont\char229}}1
	{ё}{{\"e}}1
	{ж}{{\selectfont\char230}}1
	{з}{{\selectfont\char231}}1
	{и}{{\selectfont\char232}}1
	{й}{{\selectfont\char233}}1
	{к}{{\selectfont\char234}}1
	{л}{{\selectfont\char235}}1
	{м}{{\selectfont\char236}}1
	{н}{{\selectfont\char237}}1
	{о}{{\selectfont\char238}}1
	{п}{{\selectfont\char239}}1
	{р}{{\selectfont\char240}}1
	{с}{{\selectfont\char241}}1
	{т}{{\selectfont\char242}}1
	{у}{{\selectfont\char243}}1
	{ф}{{\selectfont\char244}}1
	{х}{{\selectfont\char245}}1
	{ц}{{\selectfont\char246}}1
	{ч}{{\selectfont\char247}}1
	{ш}{{\selectfont\char248}}1
	{щ}{{\selectfont\char249}}1
	{ъ}{{\selectfont\char250}}1
	{ы}{{\selectfont\char251}}1
	{ь}{{\selectfont\char252}}1
	{э}{{\selectfont\char253}}1
	{ю}{{\selectfont\char254}}1
	{я}{{\selectfont\char255}}1
	{А}{{\selectfont\char192}}1
	{Б}{{\selectfont\char193}}1
	{В}{{\selectfont\char194}}1
	{Г}{{\selectfont\char195}}1
	{Д}{{\selectfont\char196}}1
	{Е}{{\selectfont\char197}}1
	{Ё}{{\"E}}1
	{Ж}{{\selectfont\char198}}1
	{З}{{\selectfont\char199}}1
	{И}{{\selectfont\char200}}1
	{Й}{{\selectfont\char201}}1
	{К}{{\selectfont\char202}}1
	{Л}{{\selectfont\char203}}1
	{М}{{\selectfont\char204}}1
	{Н}{{\selectfont\char205}}1
	{О}{{\selectfont\char206}}1
	{П}{{\selectfont\char207}}1
	{Р}{{\selectfont\char208}}1
	{С}{{\selectfont\char209}}1
	{Т}{{\selectfont\char210}}1
	{У}{{\selectfont\char211}}1
	{Ф}{{\selectfont\char212}}1
	{Х}{{\selectfont\char213}}1
	{Ц}{{\selectfont\char214}}1
	{Ч}{{\selectfont\char215}}1
	{Ш}{{\selectfont\char216}}1
	{Щ}{{\selectfont\char217}}1
	{Ъ}{{\selectfont\char218}}1
	{Ы}{{\selectfont\char219}}1
	{Ь}{{\selectfont\char220}}1
	{Э}{{\selectfont\char221}}1
	{Ю}{{\selectfont\char222}}1
	{Я}{{\selectfont\char223}}1
}

\newcommand{\biglisting}[1]{%
	\lstinputlisting[numbers=left]{#1}%
}

\begin{document}

\section{Постановка задачи}

\paragraph{Цель работы:} построение гистограммы и эмпирической функции распределения.

\paragraph{Содержание работы}

\begin{enumerate}
	\item Для выборки объёма $n$ из генеральной совокупности $X$ реализовать в виде программы на ЭВМ
	\begin{enumerate}
		\item вычисление максимального значения $M_{\max}$ и минимального значения $M_{\min}$;
		\item вычисление размаха $R$ выборки;
		\item вычисление оценок $\hat{\mu}$ и $S^2$ математического ожидания $\Expect X$ и дисперсии $\Variance X$;
		\item группировку значений выборки в $m = [\log_2 n] + 2$ интервала;
		\item построение на одной координатной плоскости гистограммы и графика функции плотности распределения вероятностей нормальной случайной величины с математическим ожиданием $\hat{\mu}$ и дисперсией $S^2$;
		\item построение на другой координатной плоскости графика эмпирической функции распределения и функции распределения нормальной случайной величины с математическим ожиданием $\hat{\mu}$ и дисперсией $S^2$.
	\end{enumerate}
	\item Провести вычисления и построить графики для выборки из индивидуального варианта.
\end{enumerate}
\newpage
\section{Отчет}

\subsection{Формулы для вычисления величин}

\paragraph{Количество интервалов}
\begin{equation}
m = [\log_2 n] + 2
\end{equation}


\paragraph{Минимальное значение выборки}

\begin{equation}
M_{\min} = \min \{ x_1, \dots, x_n\}, \quad \text{где}
\end{equation}
\begin{itemize}
	\item $(x_1, \dots, x_n)$ --- реализация случайной выборки.
\end{itemize}


\paragraph{Максимальное значение выборки}

\begin{equation}
M_{\max} = \max \{ x_1, \dots, x_n\}, \quad \text{где}
\end{equation}
\begin{itemize}
	\item $(x_1, \dots, x_n)$ --- реализация случайной выборки.
\end{itemize}


\paragraph{Размах выборки}

\begin{equation}
R = M_{\max} - M_{\min}, \quad \text{где}
\end{equation}
\begin{itemize}
	\item $M_{\max}$ --- максимальное значение выборки;
	\item $M_{\min}$ --- минимальное значение выборки.
\end{itemize}


\paragraph{Оценка математического ожидания}

\begin{equation}
\hat{\mu}(\vec{X}) = \overline{X} = \frac{1}{n} \sum_{i = 1}^{n} X_i\,.
\end{equation}


\paragraph{Несмещенная оценка дисперсии}

\begin{equation}
S^2(\vec{X}) = \frac{n}{n - 1}\hat{\sigma}^2(\vec{X}) = \frac{1}{n - 1}\sum_{i = 1}^{n} (X_i - \overline{X})^2\,.
\end{equation}


\subsection{Эмпирическая плотность и гистограмма}

	\emph{Эмпирической плотностью распределения  выборки $\vec{x}$} называют функцию
	\begin{equation}
	f_n(x) =
	\begin{cases}
	\frac{n_i}{n \, \Delta}, &x \in J_i,\; i = \overline{1, m};\\
	0, &\text{иначе}.
	\end{cases}, \quad \text{где}
	\end{equation}
	\begin{itemize}
		\item $J_i,\, i = \overline{1; m}$, --- полуинтервал из $J = [x_{(1)}, x_{(n)}]$, где 
		\begin{align}
		&x_{(1)} = \min\{ x_1, \dots, x_n \}, &x_{(n)} = \max\{ x_1, \dots, x_n \};
		\end{align}
		при этом все полуинтервалы, кроме последнего, не содержат правую границу т.\,е.
		\begin{align}
		&J_i = [ x_{(1)} + (i-1)\Delta, x_{(1)} + i\Delta), \quad i = \overline{1, m-1};
		\\
		&J_m = [ x_{(1)} + (m-1)\Delta, x_{(1)} + m\Delta];
		\end{align}
		\item $m$ --- количество полуинтервалов интервала $J = [x_{(1)}, x_{(n)}]$;
		\item $\Delta$ --- длина полуинтервала $J_i$, $i = \overline{1, m}$ равная
		\begin{equation}
		\Delta = \frac{x_{(n)} - x_{(1)}}{m} = \frac{|J|}{m};
		\end{equation}
		\item $n_i$ --- количество элементов выборки в полуинтервале $J_i$, $i = \overline{1, m}$;
		\item $n$ --- количество элементов в выборке.
		
	\end{itemize}

	График функции $f_n(x)$ называют \emph{гистограммой}. Гистограмма представляет собой кусочно-постоянную функцию на промежутке $J$.

\subsection{Эмпирическая функция распределения}

	\emph{Эмпирической функцией распределения, отвечающей выборке $\vec{x}$} называют функцию
	\begin{equation}
	F_n\colon \mathbb{R} \to \mathbb{R}, \qquad F_n(x) = \frac{n(x, \vec{x})}{n},
	\end{equation} 
	где $n(x, \vec{x})$ --- количество элементов выборки $\vec{x}$, которые меньше $x$.
	
\begin{rem}
    $F_n(x)$ обладает всеми свойствами функции распределения. При этом она кусочно-постоянна и принимает значения 
    \[
        0, \frac{1}{n}, \frac{2}{n}, \ldots, \frac{(n-1)}{n}, 1
    \]
\end{rem}
\begin{rem}
    Если все элементы вектора $\vec{x}_n$ различны, то
    \begin{equation}
        F_n(x) = 
        \begin{cases}
            0, & x \leq x_{(1)}; \\
            \frac{i}{n}, & x_{(i)} < x \leq x_{(i+1)},\; i = \overline{1, n-1}; \\
            1, & x > x_{(n)}.
        \end{cases}
    \end{equation}
\end{rem}
\begin{rem}
    Эмпирическая функция распределения позволяет интерпретировать выборку $\vec{x}_n$ как реализацию дискретной случайной величины $\widetilde{X}$ ряд распределения которой
    \begin{center}
        \renewcommand{\arraystretch}{1.5}
        \begin{tabular}{| c || c | c | c |}
            \hline
            $\widetilde{X}$ & $x_{(1)}$ & \ldots & $x_{(n)}$ \\
            \hline
            $\Prob$ & $1/n$ & \ldots & $1/n$ \\
            \hline
        \end{tabular}
    \end{center}
    Это позволяет рассматривать числовые характеристики случайной величины $\widetilde{X}$ как приближённые значения числовых характеристик случайной величины $X$.
\end{rem}

\newpage
\section{Листинг программы}

\begin{lstlisting}
function lab1()
    % Выборка объема n из генеральной совокупности Х
    X = [-1.12,-1.06,0.46,-0.39,0.09,-1.44,-1.64,0.86,-0.24,-1.71, ...
        -0.84,-1.19,-0.84,-0.55,-1.11,-1.84,-0.60,-0.92,-0.69,0.23, ...
        0.51,-2.41,-0.53,-1.41,-0.23,-0.89,-0.13,-1.50,0.02,0.27, ...
        -0.75,-0.06,-0.48,0.14,0.20,-2.22,-1.42,-0.54,0.83,-1.77, ...
        -0.10,-0.07,-0.94,-0.13,-1.76,-0.77,-1.26,-0.29,-1.11, ...
        -0.56,1.19,-0.92,-2.02,-1.94,-0.36,-2.09,-2.51,-1.82,0.39, ...
        -2.08,-0.60,-1.38,-1.12,-0.34,0.77,-1.34,0.24,-0.30,-1.67, ...
        -1.50,-0.77,-0.10,-0.39,-0.35,-2.23,-0.84,-0.85,-0.44,-0.20, ...
        -1.76,-0.91,-1.30,-2.03,-2.50,1.08,0.19,0.03,1.17,-0.05, ...
        -2.88,-1.13,-0.05,-1.37,-0.22,0.88,-1.04,-0.52,-1.64,-0.43, ...
        -0.09,-2.44,-0.78,-2.48,-1.16,-0.44,-0.34,-0.60,-0.11,-0.41, ...
        -0.04,-1.09,-1.81,-0.74,-1.07,-1.07,-0.68,-0.36,-0.65,-1.72, ...
        -0.49];
    
    % Минимальное значение
    Mmin = min(X);
    fprintf('Mmin = %f\n', Mmin);
    
    % Максимальное значение
    Mmax = max(X);
    fprintf('Mmax = %f\n', Mmax);
    
    % Размах выборки
    R = Mmax - Mmin;
    fprintf('R = %f\n', R);
    
    % Выборочное среднее
    mu = mean(X);
    fprintf('mu = %f\n', mu);
    
    % Несмещенная оценка дисперсии
    s2 = var(X);
    fprintf('S2 = %f\n', s2);

    % Нахождение количества интервалов
    m = floor(log2(length(X))) + 2;
    
    % Разбиваем выборку на m интервалов от min до max
    [count, edges] = histcounts(X, m, 'BinLimits', [min(X), max(X)]);
    countLen = length(count);
    
    % Интервалы и количество элементов в них
    fprintf('\nИнтервальная группировка значений выборки при m = %d \n', m);
    for i = 1 : (countLen - 1)
        fprintf('[%f : %f) - %d\n', edges(i), edges(i + 1), count(i));
    end
    fprintf('[%f : %f] - %d\n', edges(countLen), edges(countLen + 1), count(countLen));
    
    % Гистограмма
    plotHistogram(X, count, edges, m);
    hold on; 
    % График функции плотности распределения вероятностей нормальной случайной величины
    f(X, mu, s2, m, R);
    figure;
    % График эмпирической функции распределения
    plotEmpiricalF(X);
    hold on;
    % График функции распределения нормальной случайной величины
    F(sort(X), mu, s2, m, R);
    
function plotHistogram(X, count, edges, m)
    h = histogram();
    h.BinEdges = edges;
    h.BinCounts = count / length(X) / ((max(X) - min(X)) / m);
end

function f(X, MX, DX, m, R)
        delta = R/m;
        sigma = sqrt(DX);
        Xn = min(X):delta/20:max(X);
        Y = normpdf(Xn, MX, sigma);
        plot(Xn, Y, 'red');
end

function F(X, MX, DX, m, R)
        delta = R/m;
        Xn = min(X):delta/20:max(X);
        Y = 1/2 * (1 + erf((Xn - MX) / sqrt(2*DX))); 
        plot(Xn, Y, 'red');
end

function plotEmpiricalF(X)  
        [yy, xx] = ecdf(X);
        stairs(xx, yy);
end

end
\end{lstlisting}

\section{Результаты расчётов}


Mmin = -2.880000 \\
Mmax = 1.190000 \\
R = 4.070000 \\
mu = -0.771000 \\
S2 = 0.752555 \\


\begin{figure}[H]
        		\includegraphics[scale=0.8]{m}
        		\label{fig:g1}
        \end{figure}


\section{Графики}

\subsection{Гистограмма и график функции плотности распределения вероятностей нормальной случайной величины с математическим ожиданием $\hat{\mu}$ и дисперсией $S^2$}

На рис. \ref{fig:g1} представлена гистограмма и график функции плотности распределения вероятностей:
	
	\begin{figure}[H]
        	\begin{center}
        		\includegraphics[scale=0.8]{g1}
        		\caption{Гистограмма и график функции плотности распределения вероятностей}
        		\label{fig:g1}
        	\end{center}
        \end{figure}

\newpage
\subsection{График эмпирической функции распределения и функции распределения нормальной случайной величины с математическим ожиданием $\hat{\mu}$ и дисперсией $S^2$}

На рис. \ref{fig:g2} представлен график эмпирической функции распределения и функции распределения:
	
	\begin{figure}[H]
        	\begin{center}
        		\includegraphics[scale=0.8]{g2}
        		\caption{График эмпирической функции распределения и функции распределения}
        		\label{fig:g2}
        	\end{center}
        \end{figure}

\end{document}