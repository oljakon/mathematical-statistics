\documentclass[a4paper, 12pt]{article}
\usepackage[T2A]{fontenc}  
\usepackage[utf8]{inputenc} 
\usepackage[english,russian]{babel} 
\usepackage{color}    
\usepackage{float}

\usepackage[top=20mm, bottom=20mm, left = 20mm, right = 10mm]{geometry}

\usepackage{amsmath, amsfonts, amssymb, mathtools} 

\usepackage{amsthm}

\theoremstyle{definition}
\newtheorem{defn}{Определение}[section]
\newtheorem*{rem}{Замечание}
\usepackage{graphicx}

\usepackage{listings, listingsutf8}

\lstset{
	language=Matlab, 
	numbers=left,   
	frame=single, 
	escapebegin=\begin{russian}\commentfont,
    escapeend=\end{russian},
    breaklines=true,   
    breakatwhitespace=true,
    showstringspaces=false,
	literate={а}{{\selectfont\char224}}1
	{б}{{\selectfont\char225}}1
	{в}{{\selectfont\char226}}1
	{г}{{\selectfont\char227}}1
	{д}{{\selectfont\char228}}1
	{е}{{\selectfont\char229}}1
	{ё}{{\"e}}1
	{ж}{{\selectfont\char230}}1
	{з}{{\selectfont\char231}}1
	{и}{{\selectfont\char232}}1
	{й}{{\selectfont\char233}}1
	{к}{{\selectfont\char234}}1
	{л}{{\selectfont\char235}}1
	{м}{{\selectfont\char236}}1
	{н}{{\selectfont\char237}}1
	{о}{{\selectfont\char238}}1
	{п}{{\selectfont\char239}}1
	{р}{{\selectfont\char240}}1
	{с}{{\selectfont\char241}}1
	{т}{{\selectfont\char242}}1
	{у}{{\selectfont\char243}}1
	{ф}{{\selectfont\char244}}1
	{х}{{\selectfont\char245}}1
	{ц}{{\selectfont\char246}}1
	{ч}{{\selectfont\char247}}1
	{ш}{{\selectfont\char248}}1
	{щ}{{\selectfont\char249}}1
	{ъ}{{\selectfont\char250}}1
	{ы}{{\selectfont\char251}}1
	{ь}{{\selectfont\char252}}1
	{э}{{\selectfont\char253}}1
	{ю}{{\selectfont\char254}}1
	{я}{{\selectfont\char255}}1
	{А}{{\selectfont\char192}}1
	{Б}{{\selectfont\char193}}1
	{В}{{\selectfont\char194}}1
	{Г}{{\selectfont\char195}}1
	{Д}{{\selectfont\char196}}1
	{Е}{{\selectfont\char197}}1
	{Ё}{{\"E}}1
	{Ж}{{\selectfont\char198}}1
	{З}{{\selectfont\char199}}1
	{И}{{\selectfont\char200}}1
	{Й}{{\selectfont\char201}}1
	{К}{{\selectfont\char202}}1
	{Л}{{\selectfont\char203}}1
	{М}{{\selectfont\char204}}1
	{Н}{{\selectfont\char205}}1
	{О}{{\selectfont\char206}}1
	{П}{{\selectfont\char207}}1
	{Р}{{\selectfont\char208}}1
	{С}{{\selectfont\char209}}1
	{Т}{{\selectfont\char210}}1
	{У}{{\selectfont\char211}}1
	{Ф}{{\selectfont\char212}}1
	{Х}{{\selectfont\char213}}1
	{Ц}{{\selectfont\char214}}1
	{Ч}{{\selectfont\char215}}1
	{Ш}{{\selectfont\char216}}1
	{Щ}{{\selectfont\char217}}1
	{Ъ}{{\selectfont\char218}}1
	{Ы}{{\selectfont\char219}}1
	{Ь}{{\selectfont\char220}}1
	{Э}{{\selectfont\char221}}1
	{Ю}{{\selectfont\char222}}1
	{Я}{{\selectfont\char223}}1
}

\newcommand{\biglisting}[1]{%
	\lstinputlisting[numbers=left]{#1}%
}

\begin{document}

\section{Постановка задачи}

\paragraph{Цель работы:} построение доверительных интервалов для математического и дисперсии нормальной случайной величины.

\paragraph{Содержание работы:}

\begin{enumerate}
    \item Для выборки объема $n$ из нормальной генеральной совокупности $X$ реализовать в виде программы на ЭВМ
    \begin{itemize}
        \item вычисление точечных оценок $\hat{\mu}(\vec{X_n})$ и $S^2(\vec{X_n})$ математического ожидания $MX$ и дисперсии $DX$ соответственно;
        \item вычисление нижней и верхней границ $\overline{\mu}(\vec{X_n})$, $\underline{\mu}(\vec{X_n})$ для $\gamma$-доверительного интервала для математического ожидания $MX$;
        \item вычисление нижней и верхней границ $\overline{\sigma^2}(\vec{X_n})$, $\underline{\sigma^2}(\vec{X_n})$ для $\gamma$-доверительного интервала для дисперсии $DX$;
    \end{itemize}
    \item вычислить $\hat{\mu}$ и $S^2$ для выборки из индивидуального варианта;
    \item для заданного пользователем уровня доверия $\gamma$ и $N$ – объема выборки из индивидуального варианта:
    \begin{itemize}
        \item на координатной плоскости $Oyn$ построить прямую $y = \hat{\mu}(\vec{x_N})$, также графики функций $y = \hat{\mu}(\vec{x_n})$, $y = \overline{\mu}(\vec{x_n})$ и $y = \underline{\mu}(\vec{x_n})$ как функций объема $n$ выборки, где $n$ изменяется от 1 до $N$;
        \item на другой координатной плоскости $Ozn$ построить прямую $z = S^2(\vec{x_N})$, также графики функций $z = S^2(\vec{x_n})$, $z = \overline{\sigma^2}(\vec{x_n})$ и $z = \underline{\sigma^2}(\vec{x_n})$ как функций объема $n$ выборки, где $n$ изменяется от 1 до $N$.
    \end{itemize}
\end{enumerate}

\newpage
\section{Теоретическая часть}

\subsection{Определение $\gamma$-доверительного интервала для значения параметра распределения случайной величины}

Дана случайная величина X, закон распределения которой известен с точностью до неизвестного параметра $\theta$. 

Интервальной оценкой с коэффициентом доверия $\gamma$ ($\gamma$-доверительной интервальной оценкой) параметра $\theta$ называют пару статистик $\underline{\theta}(\vec X), \overline{\theta}(\vec X)$ таких, что

$$P\{\underline{\theta}(\vec X)< \theta< \overline{\theta}(\vec X)\}=\gamma$$ 

Другими словами, $\gamma$-доверительная интервальная оценка для параметра $\theta$ -- такой интервал $( \underline{\theta}(\vec X), \overline{\theta}(\vec X)\)) со случайными границами, который накрывает теоретическое (то есть "истинное") значение этого параметра с вероятностью $\gamma$.

Поскольку границы интервала являются случайными величинами, то для различных реализаций случайной выборки $\vec X$ статистики $\underline{\theta}(\vec X), \overline{\theta}(\vec X)$ могут принимать различные значения.

\subsection{Формулы для вычисления величин}

Оценка математического ожидания

$$ \hat{\mu}(\vec{X}_n) = \overline{X}_n = \frac{1}{n} \sum_{i = 1}^{n} X_i\,.$$

Несмещенная оценка дисперсии 

$$S^2(\vec{X}_n) = \frac{n}{n - 1}\hat{\sigma}^2 = \frac{1}{n - 1}\sum_{i = 1}^{n} (X_i - \overline{X}_n)^2\,.$$

Выборочная дисперсия

$$\hat{\sigma}^2(\vec{X}_n) = \frac{1}{n} \sum_{i = 1}^{n}(X_i - \overline{X}_n)^2 $$

\subsection{Формулы для вычисления границ $\gamma$- доверительного интервала для математического ожидания и дисперсии нормальной случайной величины}

Формулы для вычисления границ $\gamma$- доверительного интервала для математического ожидания:

$$
\underline\mu(\vec X_n)=\overline X - \frac{S(\vec X)t_{\frac{1+\gamma}{2}}}{\sqrt{n}}
$$

$$
\overline\mu(\vec X_n)=\overline X + \frac{S(\vec X)t_{\frac{1+\gamma}{2}}}{\sqrt{n}}
$$

$\overline X$ -- точечная оценка математического ожидания

$S^2(\vec X)$ -- точечная оценка дисперсии

$n$ -- объем выборки

$\gamma$ -- уровень доверия

$t_{\frac{1+\gamma}{2}}$ -- квантили соответствующих уровней распределения Стьюдента с $n - 1$ степенями свободы

Формулы для вычисления границ $\gamma$- доверительного интервала для дисперсии:

$$
\underline\sigma(\vec X_n)= \frac{(n-1)S^2(\vec X)}{h_{\frac{1+\gamma}{2}}}
$$

$$
\overline\sigma(\vec X_n)= \frac{(n-1)S^2(\vec X)}{h_{\frac{1-\gamma}{2}}}
$$

$S^2(\vec X)$ -- точечная оценка дисперсии

$n$ -- объем выборки

$\gamma$ -- уровень доверия

$h_{\frac{1+\gamma}{2}}$ -- квантили соответствующих уровней распределения хи-квадрат с $n - 1$ степенями свободы

\newpage
\section{Практическая часть}

\subsection{Практическая часть}

В листинге 1 представлен текст программы.

\begin{lstlisting}
function lab2()
    X = [-1.12,-1.06,0.46,-0.39,0.09,-1.44,-1.64,0.86,-0.24,-1.71,-0.84,...
        -1.19,-0.84,-0.55,-1.11,-1.84,-0.60,-0.92,-0.69,0.23,0.51,-2.41,...
        -0.53,-1.41,-0.23,-0.89,-0.13,-1.50,0.02,0.27,-0.75,-0.06,-0.48,...
        0.14,0.20,-2.22,-1.42,-0.54,0.83,-1.77,-0.10,-0.07,-0.94,-0.13,...
        -1.76,-0.77,-1.26,-0.29,-1.11,-0.56,1.19,-0.92,-2.02,-1.94,-0.36,...
        -2.09,-2.51,-1.82,0.39,-2.08,-0.60,-1.38,-1.12,-0.34,0.77,-1.34,...
        0.24,-0.30,-1.67,-1.50,-0.77,-0.10,-0.39,-0.35,-2.23,-0.84,-0.85,...
        -0.44,-0.20,-1.76,-0.91,-1.30,-2.03,-2.50,1.08,0.19,0.03,1.17,...
        -0.05,-2.88,-1.13,-0.05,-1.37,-0.22,0.88,-1.04,-0.52,-1.64,-0.43,...
        -0.09,-2.44,-0.78,-2.48,-1.16,-0.44,-0.34,-0.60,-0.11,-0.41,...
        -0.04,-1.09,-1.81,-0.74,-1.07,-1.07,-0.68,-0.36,-0.65,-1.72,-0.49];

    % Уровень доверия
    gamma = 0.9;
    
    % Объем выборки 
    n = length(X);
    
    % Точечная оценка матожидания
    mu = mean(X);
    
    % Точечная оценка дисперсии
    sigma2 = var(X);
    
    % Нижняя граница доверительного интервала для матожидания
    muLow = getMuLow(n, mu, sigma2, gamma);
    
    % Верхняя граница доверительного интервала для матожидания
    muHigh = getMuHigh(n, mu, sigma2, gamma);
    
    % Нижняя граница доверительного интервала для дисперсии
    sigma2Low = getSigma2Low(n, sigma2, gamma);
    
    % Верхняя граница доверительного интервала для дисперсии
    sigma2High = getSigma2High(n, sigma2, gamma);
    
    fprintf('mu = %.3f\n', mu);
    fprintf('S2 = %.3f\n', sigma2);
    fprintf('muLow = %.3f\n', muLow);
    fprintf('muHigh = %.3f\n', muHigh);
    fprintf('sigma2Low = %.3f\n', sigma2Low);
    fprintf('sigma2High = %.3f\n', sigma2High);
    
    muArray = zeros(1, n);
    sigma2Array = zeros(1, n);

    muLowArray = zeros(1, n);
    muHighArray = zeros(1, n);
    sigma2LowArray = zeros(1, n);
    sigma2HighArray = zeros(1, n);
    
    for i = 1 : n
        mu = mean(X(1:i));
        sigma2 = var(X(1:i));
        muArray(i) = mu;
        sigma2Array(i) = sigma2;
        muLowArray(i) = getMuLow(i, mu, sigma2, gamma);
        muHighArray(i) = getMuHigh(i, mu, sigma2, gamma);
        sigma2LowArray(i) = getSigma2Low(i, sigma2, gamma);
        sigma2HighArray(i) = getSigma2High(i, sigma2, gamma);
    end
    
    figure 
    hold on;
    plot([1,n], [mu, mu]);
    plot((1:n), muArray);
    plot((1:n), muLowArray);
    plot((1:n), muHighArray);
    xlabel('n');
    ylabel('y');
    legend('mu(x_N)','mu(x_n)','muLow(x_n)','muHigh(x_n)');
    grid on;
    hold off;
    
    figure 
    hold on;
    plot((1:n), (zeros(1, n) + sigma2));
    plot((1:n), sigma2Array);
    plot((1:n), sigma2LowArray);
    plot((1:n), sigma2HighArray);
    xlabel('n');
    ylabel('z');
    legend('sigma(x_N)','sigma(x_n)','sigmaLow(x_n)','sigmaHigh(x_n)');
    grid on;
    hold off;
    
    function muLow = getMuLow(n, mu, s2, gamma)
        muLow = mu - sqrt(s2) * tinv((1 + gamma) / 2, n - 1) / sqrt(n);
    end

    function muHigh = getMuHigh(n, mu, s2, gamma)
        muHigh = mu + sqrt(s2) * tinv((1 + gamma) / 2, n - 1) / sqrt(n);
    end

    function sigma2Low = getSigma2Low(n, s2, gamma)
        sigma2Low = ((n - 1) * s2) / chi2inv((1 + gamma) / 2, n - 1);
    end

    function sigma2High = getSigma2High(n, s2, gamma)
        sigma2High = ((n - 1) * s2) / chi2inv((1 - gamma) / 2, n - 1);
    end

end
\end{lstlisting}

\subsection{Результаты расчетов}

mu = -0.771 \\
S2 = 0.753\\
muLow = -0.902\\
muHigh = -0.640\\
sigma2Low = 0.616\\
sigma2High = 0.945\\


\subsection{Графики}

	
	\begin{figure}[H]
        	\begin{center}
        		\includegraphics[scale=0.8]{g1}
        		\caption{Оценка для математического ожидания}
        		\label{fig:g1}
        	\end{center}
        \end{figure}

	\begin{figure}[H]
        	\begin{center}
        		\includegraphics[scale=0.8]{g2}
        		\caption{Оценка для дисперсии}
        		\label{fig:g2}
        	\end{center}
        \end{figure}

\end{document}